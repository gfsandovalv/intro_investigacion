\chapter{Statement of the tentative problem}

\subsection*{Tentative title: \emph{A Comparative Analysis of Classical-limit Regime Procedures}}

The article \cite{manfrediQuantumSystemsThat1993} answers to the question \emph{¿under which conditions the evolution of a quantum system follows exactly the laws of classical dynamics?} and a very particular physical system is presented as an example, *spinless charged particle in a uniform magnetic field*. The answer to that question is given by means of Ehrenfest’s theorem, Wigner representation and Liouville equation; the potential must be a polynomial in $x$ of at most degree 2. So that the Ehrenfest’s relations are self-contained, and the Wigner equation is equivalent to the quantum Liouville equation. The same system is studied in \cite{britoParticleUniformMagnetic2007} from a slightly different perspective. Particularly, in section 4, the evolution of a wave packet solution of the system is obtained, and it is shown the centroid of such wave packets follows a classical trajectory.

However, no characterization or discussion of the classical regime beyond the conditions of those aforementioned potentials and the particular case of the charged particle is provided. In further bibliographic review, it was found that Ehrenfest’s theorem is not a fundamental statement of the classical regime, since it is neither sufficient nor necessary to characterize such a limit \cite{ballentineInadequacyEhrenfestTheorem1994}.

The intention of the present project is to provide a comparative analysis of classical-limit regime procedures in order to elucidate the \emph{how quantum systems approach to a classical behaviour?.}

In the following, sections the problem treated in each one of the articles is presented in more detail.

\subsection*{Quantum systems that follow classical dynamics \cite{manfrediQuantumSystemsThat1993}}

The Ehrenfest theorem is used to illustrate under which conditions quantum systems behave in classical fashion. The particular feature of interest for solving this inquiry is the evolution of the system. If a quantum system is said to behave classically it means that the average position and momentum evolve following classical laws \eqref{eq:Ehrenfest_force}. This condition is satisfied as long as the potential, under which the particle is subjected, is of the form $V(x, t) = a(t) + b(t)x + c(t)x^2$, i.e. a polynomial in $x$ of at most of degree 2.

It is also shown that Ehrenfest’s relations can be derived from Liouville equation [eqref], but this does not represent a fundamental property of the particular phenomenon, rather, this fact can be seen as a property of statistical descriptions of physical phenomena. Liouville's equation states that the probability density, which describes a classical ensemble, follows a continuity equation (i.e. is conserved). Within an ensemble, the expected value of a dynamical variable (an observable) is calculated using the rule ($\langle A \rangle = \int A (x, p) f(x, p, t)\text{d}x\text{d}p$)[eqref]

The Wigner representation is used for a better understanding of the connection between classical and quantum mechanics through classical statistical mechanics. In particular, for the class of potentials previously mentioned the Wigner equation [eqref] is identical to Liouville equation. This results are used for the particular case of a charged (spinless) particle subjected to a stationary uniform magnetic field. The Ehrenfest’s relations [eqref] and the Wigner representation [eqref] are obtained. In the last section of the article, the evolution of a Gaussian wave packet [wavepacket] is calculated numerically and plotted.

\section*{Particle in a uniform magnetic field under the symmetric gauge: the eigenfunctions and the time evolution of wave packets \cite{britoParticleUniformMagnetic2007}}

Here, also the charged particle evolution is studied. The author argues that there is poor bibliography on treatment of the problem by using other than Landau gauge. So the development for obtaining the wave functions and the associated eigenvalues is presented. [eqrefs] This is done by means of an algebraic method, introducing two occupation number operators. Having this treatment, it is shown that wave equations with opposite angular momentum component have the same probability density, in spite of the fact that they belong to different energy eigenvalues. A physical interpretation of this paradoxical situation will be explained in section 3, where a semiclassical development is used for interpreting it. This semiclassical is based on the interaction energy [eqref] (this idea needs more clarification)

As well as in \cite{manfrediQuantumSystemsThat1993}, the evolution is numerically calculated. This is done by evolving the wave function, which expanded in the constructed basis and with and multiplied by a Gaussian envelope, according to Scrödinger’s time-dependent equation.

\subsection*{Statement}

According to Ballantine (\cite{ballentineInadequacyEhrenfestTheorem1994,ballentineQuantumMechanicsModern2010}), Ehrenfest theorem is not sufficient condition for characterization of systems that describes a classical motion. Even more, it is not necessary since there are systems that satisfy the theorem and though does not follow classical dynamics. For the lack of sufficiency of the condition, the quantum oscillator is presented as counterexample. It satisfies Ehrenfest theorem and even though it does not present the same classical behaviour, in particular, they differ in statistical features such as thermal equilibrium energy and specific heat. (For illustrating the nonnecessity, the discussion...)

The aim of the present is to illustrate this passage to classical regime using various procedures and characterize them quantitively, mainly by comparing the evolution of averages and deviations. The procedures to be studied are corrections to the Ehrenfest \secref{sec:Ehrenfest_corrections}, quantum Hamilton-Jacobi \secref{sec:QHJ} and Wigner representation \secref{sec:Wigner_representation}. 
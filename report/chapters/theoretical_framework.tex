\chapter{Theoretical framework}

\section{Ehrenfest's Theorem}

Equations of motion in the Heisenberg pictures are (\cite{ballentineQuantumMechanicsModern2010})
\begin{align}
    \dd{Q}{t} &= \dfrac{i}{\hbar} \comm{H}{Q} = \dfrac{P}{M} \label{eq:eq_of_motion_Q}\\
    \dd{P}{t} &= \dfrac{i}{\hbar} \comm{H}{P} = F(Q) \label{eq:eq_of_motion_P}
\end{align}
where $F(Q) = -\ddp{V}{Q}$ is the force operator. 
Taking the averages values with respect to some state, 

\begin{align}
    \dd{\mean Q}{t} &= \dfrac{\mean P}{M} \\
    \dd{\mean P}{t} &= \mean {F(Q)} \label{eq:Ehrenfest_force}
\end{align}

The desired characteristic for this force, 

\begin{equation}
    \mean {F(Q)} \approx F(\mean Q) 
\end{equation}

So that the equation \eqref{eq:Ehrenfest_force} is self-contained \cite{manfrediQuantumSystemsThat1993}.

\section{Corrections to Ehrenfest Theorem}
\label{sec:Ehrenfest_corrections}
Let the deviation operators be defined as follows

\begin{align}
    \delta Q = Q - \mean Q\\
    \delta P = P - \mean P
\end{align}

Expanding equations \eqref{eq:eq_of_motion_Q} and \eqref{eq:eq_of_motion_P} around these deviation operators and averaging, relation \eqref{eq:eq_of_motion_Q} is recovered, and in place of \eqref{eq:eq_of_motion_P} now the relation reads

\begin{equation}
    \dd{p_0}{t} = F(q_0) + \dfrac{1}{2}\mean{(\delta Q)}\ddp{^2}{q_0^2} + ...
\end{equation}

where the substitutions $q_0 = \mean Q$ and $p_0 = \mean P$ are used. Any term beyond the first one are corrections to Ehrenfest's theorem.

The meaning of these terms can be understood by comparing a similar procedure for a classical ensemble. 

Let say that a classical ensemble is described by the probability density in phase space $\rho_c(q, p, t)$ which satisfies the Liouville equation,

\begin{equation}
    \left\{\ddp{}{t} + \dot q \ddp{}{q} + \dot p \ddp{}{p}\right\}\rho_c(q, p, t) = 0 
\end{equation}

The classical averages for an observable $A$ is give as follows

\begin{equation}
    A_c = \int \d{q}\,\d{p} \,\left\{A(q,p)\rho_c(q, p, t)\right \}
\end{equation}

According to this, it is found for $q_c$ and $p_c$

\begin{align}
    \dd{q_c}{t} &= \dfrac{p_c}{m} \\
    \dd{p_c}{t} &= \int \d{q}\,\d{p} \,\left\{F(q)\rho_c(q, p, t)\right \} \label{eq:classical_force}
\end{align}
Similarly, expanding equation \eqref{eq:classical_force} around $\delta q = q - q_c$, it is found 

\begin{equation}
    \dd{p_c}{t} = F(q_c) + \dfrac{1}{2}\mean{(\delta Q)}_c\ddp{^2}{q_c^2} + ...
\end{equation}

\section{Quantum Hamilton-Jacobi Equation}
\label{sec:QHJ} 
Separating an arbitrary wave equation into a real amplitude $A$ and a purely imaginary number with a phase $S$ leads to

\begin{equation}    
    \Psi(\vec x, t) = A(\vec x, t)e^{iS(\vec x, t)/\hbar}
\end{equation}

Replacing into the time-dependent Schrödinger equation, separating real and imaginary parts gives

\begin{align}
    -\dfrac{\hbar^2}{2M}\nabla^2 A - \dfrac{1}{2M}A(\vec\nabla S)^2 + VA &= -A\ddp{S}{t} \label{eq:state_phase_eq}\\
    -\dfrac{1}{2M}\left\{A\nabla^2 S + 2 (\nabla A)\cdot(\nabla S)\right\} &= \ddp{A}{t} \label{eq:state_amplitude_eq}
\end{align}
Where $V$ is the potential function.
Equation \eqref{eq:state_amplitude_eq} can be rewritten in a convenient form,

\begin{equation}
    \ddp{P}{t} + \dfrac{\nabla(P\nabla S)}{M} = 0
\end{equation}

Where the fact that the probability $P = A^2$. The equation represents a continuity equation for the probability, so it is conserved as the system evolves. 

On the other hand, equation \eqref{eq:state_phase_eq} is found to be

\begin{equation}
    \ddp{S}{t} + \dfrac{(\nabla S)^2}{2M} + V + V_Q = 0
    \label{eq:}
\end{equation}
which haves the form of a Hamilton-Jacobi equation and will be referred as Quantum Hamilton-Jacobi equation (QHJ). Here, $V_Q$ is called quantum potential and is defined as 

\begin{equation}
    V_Q = -\dfrac{\hbar^2}{2M}\dfrac{(\nabla^2 A)}{A}
\end{equation}

\section{Wigner Representation}
\label{sec:Wigner_representation}

\begin{equation}
    W(x, p, t) = \dfrac{1}{2\pi\hbar} \int \left\{\d\xi\,\exp{\left(\dfrac{i p\xi}{\hbar}\right)} \Psi^*\left(x + \dfrac{\xi}{2}, t\right) \Psi\left(x - \dfrac{\xi}{2}, t\right)\right\}
\end{equation}

This satisfies 
\begin{equation}
    \ddp{W}{t} + \dfrac{p}{m}\ddp{W}{x} = \dfrac{i}{2\pi\hbar^2} \int\d\lambda\d p^\prime \left\{\exp{\left(\dfrac{i \lambda}{\hbar}(p-p^\prime)\right)}\left[V\left(x-\dfrac{\lambda}{2}\right) - V\left(x-\dfrac{\lambda}{2}\right)\right]W(x, p^\prime, t)\right\}
\end{equation}
\section{Quantum Hamilton-Jacobi Equation}
\label{sec:QHJ}
Separating an arbitrary wave equation into a real amplitude $A$ and a purely imaginary number with a phase $S$ leads to

\begin{equation}
    \Psi(\vec x, t) = A(\vec x, t)e^{iS(\vec x, t)/\hbar}
\end{equation}

Replacing into the time-dependent Schrödinger equation, separating real and imaginary parts gives

\begin{align}
    -\dfrac{\hbar^2}{2M}\nabla^2 A - \dfrac{1}{2M}A(\vec\nabla S)^2 + VA &= -A\ddp{S}{t} \label{eq:state_phase_eq}\\
    -\dfrac{1}{2M}\left\{A\nabla^2 S + 2 (\nabla A)\cdot(\nabla S)\right\} &= \ddp{A}{t} \label{eq:state_amplitude_eq}
\end{align}
Where $V$ is the potential function.
Equation \eqref{eq:state_amplitude_eq} can be rewritten in a convenient form,

\begin{equation}
    \ddp{P}{t} + \dfrac{\nabla(P\nabla S)}{M} = 0
\end{equation}

Where the fact that the probability $P = A^2$. The equation represents a continuity equation for the probability, so it is conserved as the system evolves.

On the other hand, equation \eqref{eq:state_phase_eq} is found to be

\begin{equation}
    \ddp{S}{t} + \dfrac{(\nabla S)^2}{2M} + V + V_Q = 0
    \label{eq:}
\end{equation}
which haves the form of a Hamilton-Jacobi equation and will be referred as Quantum Hamilton-Jacobi equation (QHJ). Here, $V_Q$ is called quantum potential and is defined as

\begin{equation}
    V_Q = -\dfrac{\hbar^2}{2M}\dfrac{(\nabla^2 A)}{A}
\end{equation}
